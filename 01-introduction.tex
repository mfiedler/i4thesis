\chapter{Introduction}
%\chapter{Einleitung}
\label{sec:introduction}

general motivation for your work, context and goals: 1-2 pages

\begin{itemize}
\item \textbf{Context:} make sure to link where your work fits in
\item \textbf{Problem:} gap in knowledge, too expensive, too slow, a deficiency, superseded technology
\item \textbf{Strategy:} the way you will address the problem 
\end{itemize}


\section{Sample Section}

The following samples explain how to insert cross-references, figures and tables, how to set math, algorithms and program code, how to add references, and how to use acronyms.


\subsection{Cross References}
\label{sec:cross-ref}

Use the \verb|\label| and \verb|\cref| commands for cross references, e.g.\ to \cref{sec:cross-ref}. 

\subsection{Figures}

\Cref{fig:setup0} shows the distribution of the nodes in the sample setup at time $t=0$, as well as the initial coverage with a sensing radius of \SI{30}{\metre} and the communication graph for a communication range of \SI{50}{\metre}.
Figure captions should always be placed at the bottom of the figure.

\begin{figure}
	\centering
	\includegraphics[width=0.3\textwidth]{figures/coverage-30--0-static1}
	\includegraphics[width=0.3\textwidth]{figures/connectivity-50-0-static1}
	\caption{Coverage and connectivity for a sample replication at time $t=0$}
	\label{fig:setup0}
\end{figure}

\subsection{Subfigures}

\Crefrange{fig:setup1}{fig:setup2} show the distribution of the nodes in the sample setup at time $t=0$, as well as the initial coverage with a sensing radius of \SI{30}{\metre} and the communication graph for a communication range of \SI{50}{\metre}.

\begin{figure}%
	\centering
    \begin{subfigure}{0.3\textwidth}
      \includegraphics[width=\textwidth]{figures/coverage-30--0-static1}
      \caption{coverage}\label{fig:setup1}
    \end{subfigure}
    \begin{subfigure}{0.3\textwidth}
      \includegraphics[width=\textwidth]{figures/connectivity-50-0-static1}
      \caption{connectivity}\label{fig:setup2}
    \end{subfigure}
	\caption{Subfigures showing coverage and connectivity for a sample replication at time $t=0$}%
	\label{fig:setups12}%
\end{figure}

\subsection{Plotting}

Plotting is an important part of every evaluation.
In general, the user has to major options there: Either using an external application to generate plots, or to use an integrated, \LaTeX{}-native plotting solution.


Popular external applications include gnuplot\footnote{\url{http://www.gnuplot.info/}}, matplotlib\footnote{\url{https://matplotlib.org/stable/}} or ggplot2\footnote{\url{https://ggplot2.tidyverse.org/}}.
If using one of those, please make sure to export the plots as either \TeX{} (if available) or at least vector graphics such as PDF, and do not use bitmap graphics.

You can also make \LaTeX{} generate plots automatically using \texttt{gnuplot}.
Just put your data into a \texttt{csv} file in the \texttt{plots} subdirectory and create a \texttt{gnuplot} script ending in \texttt{.gplt}.
Follow the example from \texttt{plots/ass\_granularity.gplt}\ifthenelse{\boolean{gnuplot}}{ shown in Figure \ref{fig:AssGran}}{}.
To enable this feature, two steps are required

\begin{enumerate}
	\item Enable shell-escape in the file \verb!.latexmkrc!. See the comments there for details
	\item Pass the gnuplot option to the \verb!i4thesis! documentclass in \verb!thesis.tex!, i.e.\\\verb!\documentclass[...,gnuplot]{i4thesis}!
	\item See the example code below (hidden as long as gnuplot is disabled)
\end{enumerate}

\ifthenelse{\boolean{gnuplot}}{
\begin{figure}
  \centering
  \vspace{\gnuplotpreskip}
  \tikzset{external/export=true}
  \tikzsetnextfilename{eval-ass_granularity}
  \begin{tikzpicture}[gnuplot]
%% generated with GNUPLOT 5.0p3 (Gentoo revision r0) (Lua 5.1; terminal rev. 99, script rev. 100)
%% Do 24 Mär 2016 17:09:07 CET
\tikzset{every node/.append style={font={\footnotesize}}}
\path (-1.813,-1.110) rectangle (7.187,4.890);
\gpcolor{color=gp lt color axes}
\gpsetlinetype{gp lt axes}
\gpsetdashtype{gp dt axes}
\gpsetlinewidth{0.50}
\draw[gp path] (0.000,0.000)--(6.634,0.000);
\gpcolor{color=gp lt color border}
\gpsetlinetype{gp lt border}
\gpsetdashtype{gp dt solid}
\gpsetlinewidth{1.00}
\draw[gp path] (0.000,0.000)--(-0.125,0.000);
\node[gp node right] at (-0.309,0.000) {0.0};
\gpcolor{color=gp lt color axes}
\gpsetlinetype{gp lt axes}
\gpsetdashtype{gp dt axes}
\gpsetlinewidth{0.50}
\draw[gp path] (0.000,0.822)--(6.634,0.822);
\gpcolor{color=gp lt color border}
\gpsetlinetype{gp lt border}
\gpsetdashtype{gp dt solid}
\gpsetlinewidth{1.00}
\draw[gp path] (0.000,0.822)--(-0.125,0.822);
\node[gp node right] at (-0.309,0.822) {20.0};
\gpcolor{color=gp lt color axes}
\gpsetlinetype{gp lt axes}
\gpsetdashtype{gp dt axes}
\gpsetlinewidth{0.50}
\draw[gp path] (0.000,1.644)--(6.634,1.644);
\gpcolor{color=gp lt color border}
\gpsetlinetype{gp lt border}
\gpsetdashtype{gp dt solid}
\gpsetlinewidth{1.00}
\draw[gp path] (0.000,1.644)--(-0.125,1.644);
\node[gp node right] at (-0.309,1.644) {40.0};
\gpcolor{color=gp lt color axes}
\gpsetlinetype{gp lt axes}
\gpsetdashtype{gp dt axes}
\gpsetlinewidth{0.50}
\draw[gp path] (0.000,2.466)--(6.634,2.466);
\gpcolor{color=gp lt color border}
\gpsetlinetype{gp lt border}
\gpsetdashtype{gp dt solid}
\gpsetlinewidth{1.00}
\draw[gp path] (0.000,2.466)--(-0.125,2.466);
\node[gp node right] at (-0.309,2.466) {60.0};
\gpcolor{color=gp lt color axes}
\gpsetlinetype{gp lt axes}
\gpsetdashtype{gp dt axes}
\gpsetlinewidth{0.50}
\draw[gp path] (0.000,3.288)--(6.634,3.288);
\gpcolor{color=gp lt color border}
\gpsetlinetype{gp lt border}
\gpsetdashtype{gp dt solid}
\gpsetlinewidth{1.00}
\draw[gp path] (0.000,3.288)--(-0.125,3.288);
\node[gp node right] at (-0.309,3.288) {80.0};
\gpcolor{color=gp lt color axes}
\gpsetlinetype{gp lt axes}
\gpsetdashtype{gp dt axes}
\gpsetlinewidth{0.50}
\draw[gp path] (0.000,4.110)--(3.878,4.110);
\draw[gp path] (6.450,4.110)--(6.634,4.110);
\gpcolor{color=gp lt color border}
\gpsetlinetype{gp lt border}
\gpsetdashtype{gp dt solid}
\gpsetlinewidth{1.00}
\draw[gp path] (0.000,4.110)--(-0.125,4.110);
\node[gp node right] at (-0.309,4.110) {100.0};
\draw[gp path] (0.737,0.000)--(0.737,-0.125);
\node[gp node center] at (0.737,-0.433) {2.2};
\draw[gp path] (1.474,0.000)--(1.474,-0.125);
\node[gp node center] at (1.474,-0.433) {2.4};
\draw[gp path] (2.211,0.000)--(2.211,-0.125);
\node[gp node center] at (2.211,-0.433) {2.6};
\draw[gp path] (2.948,0.000)--(2.948,-0.125);
\node[gp node center] at (2.948,-0.433) {2.8};
\draw[gp path] (3.686,0.000)--(3.686,-0.125);
\node[gp node center] at (3.686,-0.433) {3.0};
\draw[gp path] (4.423,0.000)--(4.423,-0.125);
\node[gp node center] at (4.423,-0.433) {3.2};
\draw[gp path] (5.160,0.000)--(5.160,-0.125);
\node[gp node center] at (5.160,-0.433) {3.4};
\draw[gp path] (5.897,0.000)--(5.897,-0.125);
\node[gp node center] at (5.897,-0.433) {3.6};
\draw[gp path] (0.000,4.521)--(0.000,0.000)--(6.634,0.000)--(6.634,4.521)--cycle;
\node[gp node center,rotate=-270] at (-1.567,2.260) {Schedulable Systems[\%]};
\node[gp node center] at (3.317,-0.895) {Utilisation};
\node[gp node right] at (5.166,4.187) {ABB};
\gpfill{rgb color={0.933,0.576,0.310}} (5.350,4.110)--(6.266,4.110)--(6.266,4.264)--(5.350,4.264)--cycle;
\gpfill{rgb color={0.933,0.576,0.310}} (0.484,0.000)--(0.623,0.000)--(0.623,3.925)--(0.484,3.925)--cycle;
\gpfill{rgb color={0.933,0.576,0.310}} (1.221,0.000)--(1.360,0.000)--(1.360,3.858)--(1.221,3.858)--cycle;
\gpfill{rgb color={0.933,0.576,0.310}} (1.958,0.000)--(2.097,0.000)--(2.097,3.739)--(1.958,3.739)--cycle;
\gpfill{rgb color={0.933,0.576,0.310}} (2.695,0.000)--(2.834,0.000)--(2.834,3.570)--(2.695,3.570)--cycle;
\gpfill{rgb color={0.933,0.576,0.310}} (3.432,0.000)--(3.571,0.000)--(3.571,3.432)--(3.432,3.432)--cycle;
\gpfill{rgb color={0.933,0.576,0.310}} (4.169,0.000)--(4.308,0.000)--(4.308,3.335)--(4.169,3.335)--cycle;
\gpfill{rgb color={0.933,0.576,0.310}} (4.906,0.000)--(5.046,0.000)--(5.046,3.181)--(4.906,3.181)--cycle;
\gpfill{rgb color={0.933,0.576,0.310}} (5.644,0.000)--(5.783,0.000)--(5.783,2.878)--(5.644,2.878)--cycle;
\draw[gp path] (0.553,3.901)--(0.553,3.947);
\draw[gp path] (0.484,3.901)--(0.622,3.901);
\draw[gp path] (0.484,3.947)--(0.622,3.947);
\draw[gp path] (1.290,3.830)--(1.290,3.884);
\draw[gp path] (1.221,3.830)--(1.359,3.830);
\draw[gp path] (1.221,3.884)--(1.359,3.884);
\draw[gp path] (2.027,3.705)--(2.027,3.772);
\draw[gp path] (1.958,3.705)--(2.096,3.705);
\draw[gp path] (1.958,3.772)--(2.096,3.772);
\draw[gp path] (2.764,3.530)--(2.764,3.608);
\draw[gp path] (2.695,3.530)--(2.833,3.530);
\draw[gp path] (2.695,3.608)--(2.833,3.608);
\draw[gp path] (3.501,3.387)--(3.501,3.475);
\draw[gp path] (3.432,3.387)--(3.570,3.387);
\draw[gp path] (3.432,3.475)--(3.570,3.475);
\draw[gp path] (4.238,3.285)--(4.238,3.382);
\draw[gp path] (4.169,3.285)--(4.307,3.285);
\draw[gp path] (4.169,3.382)--(4.307,3.382);
\draw[gp path] (4.976,3.126)--(4.976,3.233);
\draw[gp path] (4.906,3.126)--(5.045,3.126);
\draw[gp path] (4.906,3.233)--(5.045,3.233);
\draw[gp path] (5.713,2.813)--(5.713,2.940);
\draw[gp path] (5.644,2.813)--(5.782,2.813);
\draw[gp path] (5.644,2.940)--(5.782,2.940);
\node[gp node right] at (5.166,3.879) {subtask};
\gpfill{rgb color={0.933,0.749,0.310}} (5.350,3.802)--(6.266,3.802)--(6.266,3.956)--(5.350,3.956)--cycle;
\gpfill{rgb color={0.933,0.749,0.310}} (0.668,0.000)--(0.807,0.000)--(0.807,3.750)--(0.668,3.750)--cycle;
\gpfill{rgb color={0.933,0.749,0.310}} (1.405,0.000)--(1.544,0.000)--(1.544,3.576)--(1.405,3.576)--cycle;
\gpfill{rgb color={0.933,0.749,0.310}} (2.142,0.000)--(2.281,0.000)--(2.281,3.329)--(2.142,3.329)--cycle;
\gpfill{rgb color={0.933,0.749,0.310}} (2.879,0.000)--(3.019,0.000)--(3.019,3.050)--(2.879,3.050)--cycle;
\gpfill{rgb color={0.933,0.749,0.310}} (3.616,0.000)--(3.756,0.000)--(3.756,2.758)--(3.616,2.758)--cycle;
\gpfill{rgb color={0.933,0.749,0.310}} (4.354,0.000)--(4.493,0.000)--(4.493,2.325)--(4.354,2.325)--cycle;
\gpfill{rgb color={0.933,0.749,0.310}} (5.091,0.000)--(5.230,0.000)--(5.230,1.791)--(5.091,1.791)--cycle;
\gpfill{rgb color={0.933,0.749,0.310}} (5.828,0.000)--(5.967,0.000)--(5.967,1.114)--(5.828,1.114)--cycle;
\draw[gp path] (0.737,3.719)--(0.737,3.779);
\draw[gp path] (0.668,3.719)--(0.806,3.719);
\draw[gp path] (0.668,3.779)--(0.806,3.779);
\draw[gp path] (1.474,3.539)--(1.474,3.611);
\draw[gp path] (1.405,3.539)--(1.543,3.539);
\draw[gp path] (1.405,3.611)--(1.543,3.611);
\draw[gp path] (2.211,3.285)--(2.211,3.371);
\draw[gp path] (2.142,3.285)--(2.280,3.285);
\draw[gp path] (2.142,3.371)--(2.280,3.371);
\draw[gp path] (2.948,3.000)--(2.948,3.098);
\draw[gp path] (2.879,3.000)--(3.018,3.000);
\draw[gp path] (2.879,3.098)--(3.018,3.098);
\draw[gp path] (3.686,2.703)--(3.686,2.811);
\draw[gp path] (3.616,2.703)--(3.755,2.703);
\draw[gp path] (3.616,2.811)--(3.755,2.811);
\draw[gp path] (4.423,2.263)--(4.423,2.384);
\draw[gp path] (4.354,2.263)--(4.492,2.263);
\draw[gp path] (4.354,2.384)--(4.492,2.384);
\draw[gp path] (5.160,1.728)--(5.160,1.853);
\draw[gp path] (5.091,1.728)--(5.229,1.728);
\draw[gp path] (5.091,1.853)--(5.229,1.853);
\draw[gp path] (5.897,1.054)--(5.897,1.172);
\draw[gp path] (5.828,1.054)--(5.966,1.054);
\draw[gp path] (5.828,1.172)--(5.966,1.172);
\node[gp node right] at (5.166,3.571) {task};
\gpfill{rgb color={0.263,0.329,0.635}} (5.350,3.494)--(6.266,3.494)--(6.266,3.648)--(5.350,3.648)--cycle;
\gpfill{rgb color={0.263,0.329,0.635}} (0.852,0.000)--(0.991,0.000)--(0.991,2.161)--(0.852,2.161)--cycle;
\gpfill{rgb color={0.263,0.329,0.635}} (1.589,0.000)--(1.729,0.000)--(1.729,1.597)--(1.589,1.597)--cycle;
\gpfill{rgb color={0.263,0.329,0.635}} (2.327,0.000)--(2.466,0.000)--(2.466,1.126)--(2.327,1.126)--cycle;
\gpfill{rgb color={0.263,0.329,0.635}} (3.064,0.000)--(3.203,0.000)--(3.203,0.647)--(3.064,0.647)--cycle;
\gpfill{rgb color={0.263,0.329,0.635}} (3.801,0.000)--(3.940,0.000)--(3.940,0.338)--(3.801,0.338)--cycle;
\gpfill{rgb color={0.263,0.329,0.635}} (4.538,0.000)--(4.677,0.000)--(4.677,0.205)--(4.538,0.205)--cycle;
\gpfill{rgb color={0.263,0.329,0.635}} (5.275,0.000)--(5.414,0.000)--(5.414,0.081)--(5.275,0.081)--cycle;
\gpfill{rgb color={0.263,0.329,0.635}} (6.012,0.000)--(6.151,0.000)--(6.151,0.024)--(6.012,0.024)--cycle;
\draw[gp path] (0.921,2.108)--(0.921,2.212);
\draw[gp path] (0.852,2.108)--(0.990,2.108);
\draw[gp path] (0.852,2.212)--(0.990,2.212);
\draw[gp path] (1.659,1.545)--(1.659,1.646);
\draw[gp path] (1.589,1.545)--(1.728,1.545);
\draw[gp path] (1.589,1.646)--(1.728,1.646);
\draw[gp path] (2.396,1.078)--(2.396,1.171);
\draw[gp path] (2.327,1.078)--(2.465,1.078);
\draw[gp path] (2.327,1.171)--(2.465,1.171);
\draw[gp path] (3.133,0.607)--(3.133,0.684);
\draw[gp path] (3.064,0.607)--(3.202,0.607);
\draw[gp path] (3.064,0.684)--(3.202,0.684);
\draw[gp path] (3.870,0.308)--(3.870,0.366);
\draw[gp path] (3.801,0.308)--(3.939,0.308);
\draw[gp path] (3.801,0.366)--(3.939,0.366);
\draw[gp path] (4.607,0.181)--(4.607,0.227);
\draw[gp path] (4.538,0.181)--(4.676,0.181);
\draw[gp path] (4.538,0.227)--(4.676,0.227);
\draw[gp path] (5.344,0.065)--(5.344,0.094);
\draw[gp path] (5.275,0.065)--(5.413,0.065);
\draw[gp path] (5.275,0.094)--(5.413,0.094);
\draw[gp path] (6.081,0.015)--(6.081,0.032);
\draw[gp path] (6.012,0.015)--(6.150,0.015);
\draw[gp path] (6.012,0.032)--(6.150,0.032);
\draw[gp path] (0.000,4.521)--(0.000,0.000)--(6.634,0.000)--(6.634,4.521)--cycle;
%% coordinates of the plot area
\gpdefrectangularnode{gp plot 1}{\pgfpoint{0.000cm}{0.000cm}}{\pgfpoint{6.634cm}{4.521cm}}
\end{tikzpicture}
%% gnuplot variables

  \tikzset{external/export=false}
  \vspace{\gnuplotcaptionskip}
  \caption{Prevalent assignment and scheduling on \emph{task} or \emph{subtask-level} compared to scheduling at the granularity of ABB for the same set of 12,812 random systems with up to 68 ABB.}
  \label{fig:AssGran}
  \vspace{\gnuplotundercaptionskip}
\end{figure}
}{}

Finally, the pgfplots\footnote{\url{http://pgfplots.sourceforge.net/gallery.html}} package constitutes a fully \LaTeX{}-native solution to plot measurement data, and thus does not require external applications or the shell-escape feature. As this plotting happens in the context of your current document, it ensures a consistent styling and allows you to directly annotate datapoints by using Ti\textit{k}Z. However, due to the limits of the \TeX{}-engine, it might struggle with particularly large datasets.

\subsection{Tables}

\Cref{tab:SensorNetworkApplications} gives an overview of the discussed application classes.
In contrast to figure captions, table captions are always placed above the table.

\begin{table}
	\centering
	\caption{Sensor network applications}
	\label{tab:SensorNetworkApplications}
	\begin{tabular}{>{\raggedright}p{1.8cm}p{5.4cm}p{3.4cm}}
		\toprule
		Class & application examples & lifetime aspects \\
		\midrule
		Critical, coverage & 
				Forest fire detection, flood detection, nuclear/chemical/biological attack detection, battlefield surveillance, intrusion detection & 
				$c_{ca}$/$c_{ct}$/$c_{cb}$, $c_{ln}$, $c_{la}$, $c_{lo}$\\
		Critical, no coverage & 
				Monitoring human physiological data, military monitoring of friendly forces, machine monitoring & 
				$c_{cc}$, $c_{ln}$, $c_{la}$, $c_{lo}$ \\
		Noncritical, coverage & 
				Agriculture, smart buildings, habitat monitoring (sensors monitor the inhabitants in a region) & 
				$c_{ac}$/$c_{tc}$/$c_{bc}$, $c_{cc}$, $c_{sd}$ \\
		Noncritical, no coverage & 
				Home automation, habitat monitoring (sensors are attached to animals and monitor their health and social contacts) & 
				$c_{cc}$, $c_{sd}$ 	\\
		\bottomrule
	\end{tabular}
\end{table}


\subsection{Math}

Simple inlined equations: $\zeta(t) = \min( \zeta_{**}(t))$.
The same in a numbered equation, i.e.\ \cref{eq:zeta}:

\begin{equation}
\zeta(t) = \min( \zeta_{**}(t))
\label{eq:zeta}
\end{equation}

Equations covering multiple lines should be aligned. Note that the numbering is added automatically, independent of whether the equation is actually referenced or not:

\begin{align}
sd_{max} &= max((t_{i+1} - t_i) : \zeta(t_i) < 1, i \in [0, |T|-1]) \\
\psi_{sd}(t) &= \left\{ \begin{array}{cl}
\dfrac{\Delta t_{sd}}{sd_{max}} & sd_{max} > 0 \\
1 & sd_{max} = 0 \\
\end{array} \right.\\
\zeta_{sd}(t) &= \frac{\psi_{sd} - cl_{sd}}{c_{sd} - cl_{sd}}
\end{align}


\subsection{Units}

Units should be set using the \verb|\SI| command: the measurements show that the car was accelerating at \SI{5}{\metre\per\second\squared} until it reached its final speed of \SI{100}{\kilo\metre\per\hour}. Longer unitless numbers or ranges can be typeset using the \verb|\num| and \verb|\numrange| commands, respectively: The number \num{12345678} lies in the range of \numrange{10000000}{20000000}. The value \SI{4.1(2)}{\meter} is not exactly measured. \Cref{tab:si-in-tables} gives an example of how to typeset numbers and units in tables.
\begin{table}
	\centering
	\caption{EMIT factors for a category 9 vehicle}
	\label{tab:si-in-tables}
	\begin{tabular}{l>{\raggedright}p{4cm}S[table-text-alignment=left,table-format=1.4e-1]s}
	\toprule
		\multicolumn{2}{l}{factor} & \multicolumn{1}{l}{value} & \multicolumn{1}{c}{unit} \\
	\midrule
		$M$ & vehicle mass & 1.3250e+3 & \kilo\gram \\
		$g$ & gravitational constant & 9.81 & \metre\per\second\squared \\
		$\vartheta$ & road grade & 0 & \degree \\
 		$\alpha$ & & 1.1100 & \gram\per\second \\
 		$\delta$ & & 1.9800e-6 & \gram\per\meter\cubed\second\squared \\
	\bottomrule
	\end{tabular}
\end{table}

\subsection{Algorithms}

Based on the periodically transmitted \texttt{hello} messages, the joining node gets information about its physical neighbors and their adjacent nodes. \Cref{alg:H_hello} depicts the handling of \texttt{hello} messages.

\begin{algorithm}
\begin{algorithmic}[1]
\REQUIRE Locally stored state of all neighbors in set $N$
\ENSURE Maintain neighbor set $N$ and set virtual address
\STATE Receive neighbor information from node $N_i$
\IF {$N_i \notin N$}
	\STATE $N \gets N_i$
\ELSE
	\STATE Update $N_i \in N$
\ENDIF
\IF {$P==-1$ AND (Time() $-$ OldTime) $> T_{ps}$}
	\STATE OldTime $\gets$ Time()
	\STATE SetMyPosition()
\ENDIF
\end{algorithmic}
\caption{Handle \texttt{hello} messages}
\label{alg:H_hello}
\end{algorithm}

\subsection{Program Code}

Program code should be omitted, but if absolutely necessary, it should be set as seen in \cref{lst:code}.

\begin{lstlisting}[style=txt,caption=Sample application,label=lst:code]{}
APPLICATION("printU", 192, arg)
{
    // Set Priority
    NutThreadSetPriority(16);
    // main() loop
    for (;;) {
        putchar('U');
        NutSleep(125);
    }
}
\end{lstlisting}

\subsection{References}

To further evaluate the applicability of our definition, we analyzed sensor network applications as surveyed in~\cite{akyildiz2002survey,arampatzis2005survey,khemapach2005survey}. Concerning the importance of different lifetime criteria, most of the application scenarios can be grouped into two main classes with two sub-classes each~\cite{dietrich2009lifetime}.


\subsection{Acronyms}

Acronyms should be explained when first used. Latex helps, e.g.\ \acp{MANET} have been frequently used as examples for the development of \ac{WSN} applications.

\subsection{References to Data}

With dataref, \LaTeX{}
provides a package to annotate data symbolically within the text. The
data is declared in \verb|data.tex| and can be used with the
\verb|\dref| macro and its companions. See the dataref documentation
for further examples.

\begin{quote}
  We concluded \dref{/experiment 1/base}
  experiments. \drefrel[prefix=/experiment 1,base=/base,factor,percent]{/a} percent of all experiments
  were successful.
\end{quote}

\subsection{TODOs and FIXMEs}

You can use the \verb|\TODO| command to add short ``sticky notes'' to your document. 
\TODO{This is what a TODO looks like}
This will also trigger generation of a list-of-TODOs at the end of the document. 
The same goes for the \verb|\FIXME| command.\FIXME{This is what a FIXME looks like}
Be careful when using \verb|\TODO| or \verb|\FIXME| near figures when using the gnuplot feature.

\subsection{TikZ}

\begin{quote}
  TikZ ist kein Zeichenprogramm
\end{quote}

\begin{figure}[h]\centering
\begin{tikzpicture}
  \node[transform shape,scale=10,inner sep=0] (left) {A};

  \node[transform shape,scale=10, % Scale the whole node
        right=3 of left,          % Placement of the node
        draw,                     % Draw a border
        rounded corners,          % Give it some rounded corners
        pattern=bricks,pattern color=red!50,          % Best pattern available!
        label={[rotate=90,anchor=west]north:LABEL},   % Each label is a node
        ]
     (right) % Name of Node
     {B};    % Text within Node
  
  \draw[->] (left) -- (right); 

  \draw[->,ultra thick] (left.east) to[out=20,in=200] (right.base west); 

\end{tikzpicture}
\caption{This is a useful caption for a very useful TikZ picture}
\end{figure}
